%This imports the file containing all of the configuration for the document
\documentclass [10pt]{article}
\oddsidemargin=0.0cm
\textwidth=16.5cm
\textheight=22.8cm
\topmargin=-1.0cm
\usepackage[utf8]{inputenc}
\usepackage{booktabs}
\usepackage{fancyhdr}
\usepackage{multicol}
\usepackage{mathtools}% Loads amsmath
\usepackage{lib/flowchart}
\pagestyle{fancy}
\fancyhf{}
\renewcommand{\headrulewidth}{1pt}
\usepackage{listings}

% the following is needed for syntax highlighting
\usepackage{color}
\usepackage{lib/arm} %arm keywords for syntax highlighting
\definecolor{dkgreen}{rgb}{0,0.6,0}
\definecolor{gray}{rgb}{0.5,0.5,0.5}
\definecolor{mauve}{rgb}{0.58,0,0.82}

\lstset{ %
language=[ARM]{Assembler},      % the language of the code
basicstyle=\scriptsize,         % the size of the fonts that are used for the code
numbers=left,                   % where to put the line-numbers
numberstyle=\tiny\color{gray},  % the style that is used for the line-numbers
stepnumber=1,                   % the step between two line-numbers. If it's 1, each line
% will be numbered
numbersep=5pt,                  % how far the line-numbers are from the code
backgroundcolor=\color{white},  % choose the background color. You must add \usepackage{color}
showspaces=false,               % show spaces adding particular underscores
showstringspaces=false,         % underline spaces within strings
showtabs=false,                 % show tabs within strings adding particular underscores
frame=single,                   % adds a frame around the code
rulecolor=\color{white},        % if not set, the frame-color may be changed on line-breaks within not-black text (e.g. commens (green here))
tabsize=2,                      % sets default tabsize to 2 spaces
captionpos=b,                   % sets the caption-position to bottom
breaklines=true,                % sets automatic line breaking
breakatwhitespace=false,        % sets if automatic breaks should only happen at whitespace
title=\lstname,                 % show the filename of files included with \lstinputlisting;
% also try caption instead of title
keywordstyle=\color{blue},          % keyword style
commentstyle=\color{gray},       % comment style
stringstyle=\color{mauve},         % string literal style
escapeinside={\%*}{*)},            % if you want to add a comment within your code
morekeywords={*,...}               % if you want to add more keywords to the set
}

\usepackage{tikz}
\usetikzlibrary{fit,arrows,calc,positioning}

%header and Footer set up
\lhead{daw24} %, UBIT\_2}% Left Heading(Must by you and your partners UBIT name)
\rhead{Lab 7: Q'bert}% Right Heading(Must be the title of the lab)
\rfoot{Page \thepage}% Right Footer - Page numbers
\lfoot{R1}% Left Footer

% Start the document
\begin{document}
  %%%%%%%%%%%%%%%%%%%%%%%
  %START OF THE DOCUMENT%
  %%%%%%%%%%%%%%%%%%%%%%%

  %Auto Generates the table of contents
  \tableofcontents\newpage

  %%%%%%%%%%
  %OVERVIEW%
  %%%%%%%%%%

  %Section of document will be labeled numerically ex. 1,2,3.. for each time it is used
\section{Lab 7 Overview}
  Lab 7 is a version of the video game Q'bert(Gottlieb 1982) written in ARM
  assembly language. Q'bert is depicted by the character 'Q' on a two 
  dimmensional pyramid. Three types of enemies can be on the pyramid with
  Q'bert. Balls and snake balls behave the same with the exception that the 
  snake ball becomes a snake when it reaches the bottom of the pyramid. There
  will not be more than two balls, and 1 snake on the pyramid at a time. Each
  time Q'bert moves to a new square it is cleared of '///' characters and awards
  the player points. Once all squares have been cleared the game begins again on
  a new level with the uncleared squares and an increased game speed. Q'bert
  begins the game with four lives. A life is lost whenever Q'bert jumps off the
  pyramid or occupies the same square as a ball or snake. The game ends after two
  minutes of gameplay. The objective of the game is for the player to score as
  many points before lives are lost or the game ends due to two minute timer.
  Ten points are awarded for each square cleared, and 100 points for each level
  reached. The player also recieves 25 bonus points for each life remaining at
  time out.
  \subsection{Lab 7 Usage}
   The UART, GPIO, and interrupts are all initialized when the game is started.
   The puTTY terminal is cleared, a 0 is displayed on the seven-segment display,
   and the RGB LED is set to white. An intro with instructions on how to play 
   the game is written to the cleared puTTY terminal. The program then waits for 
   interrupts, specifically a 'g' key press to start a new game, or a space bar 
   press to quit the game. 
  \subsection{Division of Work}
   David was the sole member for this lab, and is responsible for all code and documentation. 
\newpage
  \subsection{Lab 7 Flow Chart}
  \begin{center}
    \begin{tikzpicture}[node distance=2cm, scale=.50, transform shape]
      \node (0) [startstop] {start};
      \node (1) [process, below of=0] {Setup UART0};
      \node (2) [process, below of=1] {Setup GPIO};
      \node (3) [process, below of=2] {Setup Interrupts};
      \node (4) [process, below of=3] {Clear puTTY screen};
      \node (5) [process, below of=4] {Display 0 on 7-seg};
      \node (6) [process, below of=5] {Set RGB to white};
      \node (7) [process, below of=6] {Write intro/instructions to puTTY};
      \node (8) [decision, below of=7, yshift=-2cm] {Handle Interrupt};
      \node (9) [startstop, below of=8, yshift=-2cm] {Stop};
      %\node (10) [process, right of=8, xshift=3cm] {New game};      

	\draw [arrow] (0) -- (1);
      \draw [arrow] (1) -- (2);
      \draw [arrow] (2) -- (3);
      \draw [arrow] (3) -- (4);
      \draw [arrow] (4) -- (5);
      \draw [arrow] (5) -- (6);
      \draw [arrow] (6) -- (7);
      \draw [arrow] (7) -- (8);
	\draw [arrow] (8) -- node[anchor=west]{Space Bar}(9);
	%\draw [arrow] (8) -- node[anchor=south]{'g'}(10);
	\draw [arrow] (8) -| ([xshift=2cm]8.east) |- (8.north);      
    \end{tikzpicture}
  \end{center}

  \newpage
  %%%%%%%%%%%%%
  %SUBROUTINES%
  %%%%%%%%%%%%%
  \noindent
  \section{Subroutines}
  All subroutines are stored in the library file. Each routine is 
  imported at the beginning of the file.\\\\ %"\\" skips a line

  %%\begin{lstlisting}
    \begin {enumerate}
	\item{} newGame
        \item{} gameOver
	\item{} resetAllSquares
	\item{} increaseLevel
	\item{} removeQ
	\item{} redrawQ
	\item{} FIQ\_Handler
	\item{} pauseGame
	\item{} handleMoves
	\item{} spawnEnemy
	\item{} moveBall1
	\item{} moveBall2
	\item{} moveSnakeBall
	\item{} moveSnake
	\item{} randomNum

	
    \end {enumerate}
  %%\end{lstlisting}

\newpage
  \subsection{newGame Routine}
   \indent
    This routine starts a new game. All values are reset for the new game.
    This routine is predominately intended to be used from the intro/instruction
    screen, or the game over screen but can be used at anytime the game is not
    paused.
   \vskip 8pt
   \noindent
   {\bf Usage }\\
    The newGame routine first clears the screen of what is currently 
    being displayed. The score us reset to zero. The number of enemies
    currently on the pyramid is set to zero. All squares are set to 
    uncleared. The level is set to one, and the lives are set to four.
    A one is displayed on the seven-segment display to illustrate the 
    current level. All four LEDs are turned on to represent that the 
    player has four lives left. The RGB LED is set to green to show
    that the game is currently running. Q'bert's('Q') position is set
    to the starting position at the top of the pyramid, and the game
    board is drawn with 'Q' at top. Timer0 and timer1 are enabled(started).
 \vskip 8pt
  \noindent

%\newpage
  \subsubsection{newGame Flow Chart}
  \begin{center}
    \begin{tikzpicture}[node distance=2cm, scale=.50, transform shape]
      \node (0) [startstop] {start};
      \node (1) [process, below of=0] {Clear Screen};
      \node (2) [process, below of=1] {Reset score, enemies, squares, level, and lives};
      \node (3) [process, below of=2] {Display 1 on 7-seg};
      \node (4) [process, below of=3] {Turn on all four LEDs};
      \node (5) [process, below of=4] {Set RGB LED to green};
      \node (6) [process, below of=5] {Reset Q to starting position};
      \node (7) [process, below of=6] {Draw the gameboard and 'Q'};
      \node (8) [process, below of=7] {Enable timer0 and timer1};
      \node (9) [startstop, below of=8] {Stop};

      \draw [arrow] (0) -- (1);
      \draw [arrow] (1) -- (2);
      \draw [arrow] (2) -- (3);
      \draw [arrow] (3) -- (4);
      \draw [arrow] (4) -- (5);
      \draw [arrow] (5) -- (6);
      \draw [arrow] (6) -- (7);
      \draw [arrow] (7) -- (8);
      \draw [arrow] (8) -- (9);
    \end{tikzpicture}
  \end{center}


\newpage
  \subsection{gameOver Routine}
   \indent
     Displays the game over text when two minute timer is up or
     when all lives are lost. The player my either press 'g' for 
     a new game, or space bar to quit.
   \vskip 8pt
   \noindent
   {\bf Usage }\\
     This routine clears the current puTTY display and shows the 
     game over text. The number of lives is multiplied by 25 and
     added to the current score. This final score is then displayed
     at the top of the screen. Both timers are disabled. The
     IS\_GAMEOVER\_SCREEN is set. This is checked by other routines
     in the FIQ handler to prevent certain key interrupts during the
     game over screen.
 \vskip 8pt
  \noindent

%\newpage
  \subsubsection{gameOver Flow Chart}
  \begin{center}
    \begin{tikzpicture}[node distance=2cm, scale=.50, transform shape]
      \node (0) [startstop] {start};
      \node (1) [process, below of=0] {Clear Screen};
      \node (2) [process, below of=1] {Display Game Over text};
      \node (3) [process, below of=2] {SCORE += 4 * LIVES};
      \node (4) [process, below of=3] {Update Score on screen};
      \node (5) [process, below of=4] {Disable timer0 and timer1};
      \node (6) [process, below of=5] {Set IS\_GAMEOVER\_SCREEN};
      \node (7) [startstop, below of=6] {Stop};

      \draw [arrow] (0) -- (1);
      \draw [arrow] (1) -- (2);
      \draw [arrow] (2) -- (3);
      \draw [arrow] (3) -- (4);
      \draw [arrow] (4) -- (5);
      \draw [arrow] (5) -- (6);
      \draw [arrow] (6) -- (7);
    \end{tikzpicture}
  \end{center}

\newpage
  \subsection{resetAllSquares Routine}
   \indent
     Squares are set when Q'bert has explored them. They need to be 
     cleared at the start of a new game, or when proceeding to the
     the next level. 
   \vskip 8pt
   \noindent
   {\bf Usage }\\
 	A coutner register is set to zero. The first square DCD address
 	is loaded. The value in that address is cleared. Move to next. 
        Store a zero and increment counter by one. Check if the counter
        is equal to 22, and move to next address if not.
 \vskip 8pt
  \noindent

%\newpage
  \subsubsection{resetAllSquares Flow Chart}
  \begin{center}
    \begin{tikzpicture}[node distance=2cm, scale=.50, transform shape]
        \node (0) [startstop] {start};
	\node (1) [process, below of=0] {Set counter register to zero};
	\node (2) [process, below of=1] {Load square DCD address};
	\node (3) [process, below of=2] {Store a zero};
	\node (4) [process, below of=3] {Load square DCD address plus counter, LSL 2};
	\node (5) [process, below of=4] {Store a zero};
	\node (6) [process, below of=5] {Incremenet counter};
        \node (7) [decision, below of=6, yshift=-2cm] {Is remainder 22?};
        \node (8) [startstop, below of=7, yshift=-1cm] {Stop};
    
        \draw[arrow] (0) -- (1);
        \draw[arrow] (1) -- (2);
        \draw[arrow] (2) -- (3);
        \draw[arrow] (3) -- (4);
        \draw[arrow] (4) -- (5);
        \draw[arrow] (5) -- (6);
        \draw[arrow] (6) -- (7);
        \draw[arrow] (7) -- node [anchor=west] {yes} (8);
        \draw[arrow] (7) -- node [anchor=south] {no} ([xshift=-4cm]7.west) |- (4);
    \end{tikzpicture}
  \end{center}

\newpage
  \subsection{increaseLevel Routine}
   \indent
    The level is increased to the next one whenever the player explores and
    clears all the squares on the pyramid. Q'bert is moved back to the starting
    positon, and all enemies are reset. The speed of the game is increased with each
    level, not to exceed two player, or one enemy, moves every 0.1 seconds.   
   \vskip 8pt
   \noindent
   {\bf Usage }\\
     The LEVEL is increased by one. The new value is displayed on the 7-segment
     display. One-hundred points are added to the score. Q'bert's X and Y positions
     are set to the starting square at the top of the pyramid. BALL1\_SQUARE, 
     BALL2\_SQUARE, SNAKEBALL\_SQUARE, and SNAKE\_SQUARE are all set to zero.
     The screen is cleared and a new GAME\_BOARD is displayed. INC\_TIMER\_FLAG is
     set and NUM\_BALLS is cleared to zero. 
 \vskip 8pt
  \noindent

%\newpage
  \subsubsection{inreaseLevel Flow Chart}
  \begin{center}
    \begin{tikzpicture}[node distance=2cm, scale=.50, transform shape]
        \node (0) [startstop] {start};
	\node (1) [process, below of=0] {LEVEL = LEVEL + 1};
	\node (2) [process, below of=1] {display current level on 7-seg};
	\node (3) [process, below of=2] {SCORE = SCORE + 100};
	\node (4) [process, below of=3] {REST Q\_X\_POSITION to 15};
	\node (5) [process, below of=4] {REST Q\_Y\_POSITION to 5};
	\node (6) [process, below of=5] {Clear BALL1\_SQUARE};
	\node (7) [process, below of=6] {Clear BALL2\_SQUARE};
	\node (8) [process, below of=7] {Clear SNAKEBALL\_SQUARE};
	\node (9) [process, below of=8] {Clear SNAKE\_SQUARE};
	\node (10) [process, below of=9] {resetAllSquares};
	\node (11) [process, below of=10] {Clear screen};
	\node (12) [process, below of=11] {Redraw gameboard};
	\node (13) [process, below of=12] {set INC\_TIMER\_FLAG};
	\node (14) [process, below of=13] {Clear NUM\_BALLS};
        \node (15) [startstop, below of=14, yshift=-1cm] {Stop};
    
        \draw[arrow] (0) -- (1);
        \draw[arrow] (1) -- (2);
        \draw[arrow] (2) -- (3);
        \draw[arrow] (3) -- (4);
        \draw[arrow] (4) -- (5);
        \draw[arrow] (5) -- (6);
        \draw[arrow] (6) -- (7);
        \draw[arrow] (7) -- (8);
        \draw[arrow] (8) -- (9);
        \draw[arrow] (9) -- (10);
        \draw[arrow] (10) -- (11);
        \draw[arrow] (11) -- (12);
        \draw[arrow] (12) -- (13);
        \draw[arrow] (13) -- (14);
        \draw[arrow] (14) -- (15);
    \end{tikzpicture}
  \end{center}

\newpage
  \subsection{removeQ Routine}
   \indent
	Replace 'Q' on the screen with a ' '(space).
   \vskip 8pt
   \noindent
   {\bf Usage }\\
	Use ansi excape secquence, ESC[\#;\#f to move to line \#, and
   	column \#. Output a ' ' space to overwite the 'Q' that is 
	currently there. 
 \vskip 8pt
  \noindent

%\newpage
  \subsubsection{removeQ Flow Chart}
  \begin{center}
    \begin{tikzpicture}[node distance=2cm, scale=.50, transform shape]
        \node (0) [startstop] {start};
	\node (1) [process, below of=0] {Use ansi escape sequence to move cursor};
	\node (2) [process, below of=1] {Overwrite 'Q' with a ' '(space)};
        \node (3) [startstop, below of=2] {Stop};
    
        \draw[arrow] (0) -- (1);
        \draw[arrow] (1) -- (2);
        \draw[arrow] (2) -- (3);
    \end{tikzpicture}
  \end{center}

\newpage
  \subsection{redrawQ Routine}
   \indent
     Routine that draws Q'bert ('Q') in his new position of the pyramid based
     on key inputs from the player. Called twice every second for the first level,
     twice every 0.9 seconds for the second level and 0.1 seconds less for every
     level after. It will stop decrasing at 0.1 seconds.
   \vskip 8pt
   \noindent
   {\bf Usage }\\
	Check which direction Q'bert will be moved. Adjust Q'bert's X and Y 
	position and update his square to refelect new position. Clear unexplored
	square and check if all squares have been explored. Incease level if all have
 	been explored.  
 \vskip 8pt
  \noindent

\newpage
  \subsubsection{redrawQ Flow Chart}
  \begin{center}
    \begin{tikzpicture}[node distance=2cm, scale=.50, transform shape]
        \node (0) [startstop] {start};
        \node (1) [decision, below of=0, yshift=-1cm] {Q\_direction = up?};
        \node (2) [decision, below of=1, yshift=-4cm] {Q\_direction = down?};
        \node (3) [decision, below of=2, yshift=-4cm] {Q\_direction = right?};
        \node (4) [decision, below of=3, yshift=-4cm] {Q\_direction = left?};
        \node (5) [process, below of=4, yshift=-2cm] {Use ansi escape sequence to move cursor};
        \node (6) [process, below of=5, yshift=-2cm] {Output a 'Q'};
        \node (7) [process, below of=6, yshift=-2cm] {updateSquare};
        \node (8) [decision, below of=7, yshift=-2cm] {All squares cleared?};
        \node (9) [startstop, below of=8, yshift=-1cm] {Stop};
        \node (21) [process, left of=2, xshift=-4cm] {Q\_X\_POSIYION -= 2};
	\node (22) [process, below of =21] {Q\_Y\_POSITION += 4};
        \node (23) [process, left of=4, xshift=-4cm] {Q\_X\_POSITION += 5};
        \node (24) [process, below of =23] {Q\_Y\_POSITION += 2};
        \node (25) [process, left of=8, xshift=-4cm] {increaseLevel};	
        \node (31) [process, right of=1, xshift=4cm] {Q\_X\_POSIYION += 2};
	\node (32) [process, below of =31] {Q\_Y\_POSITION -= 4};
        \node (33) [process, right of=3, xshift=4cm] {Q\_X\_POSITION -=5};
        \node (34) [process, below of =33] {Q\_Y\_POSITION -= 2};
	
        \draw[arrow] (0) -- (1);
        \draw[arrow] (1) -- node [anchor=west] {no} (2);
        \draw[arrow] (2) -- node [anchor=west] {no} (3);
        \draw[arrow] (3) -- node [anchor=west] {no} (4);
        \draw[arrow] (4) -- node [anchor=south] {no} ([xshift=2cm]4.east) |- (9);
        \draw[arrow] (5) -- (6);
        \draw[arrow] (6) -- (7);
        \draw[arrow] (7) -- (8);
        \draw[arrow] (8) -- node [anchor=west] {no} (9);
        \draw[arrow] (31) -- (32);
        \draw[arrow] (33) -- (34);
        \draw[arrow] (8) -- node [anchor=south] {yes} (25);
        \draw[arrow] (1) -- node [anchor=south] {yes} (31);
        \draw[arrow] (2) -- node [anchor=south] {yes} (21);
        \draw[arrow] (3) -- node [anchor=south] {yes} (33);
        \draw[arrow] (4) -- node [anchor=south] {yes} (23);
        \draw[arrow] (21) -- (22);
        \draw[arrow] (22) -- ([xshift=-1cm]22.west) |- (5);
        \draw[arrow] (24) |- (5);
        \draw[arrow] (31) -- ([xshift=1cm]31.east) |- (5);
        \draw[arrow] (34) |- (5);
        \draw[arrow] (23) -- (24);
        \draw[arrow] (25) |- (9);
    \end{tikzpicture}
  \end{center}

\newpage
  \subsection{removeLife Routine}
   \indent
	This routine doubles the speed that the character moves across
	the puTTy terminal. 
   \vskip 8pt
   \noindent
   {\bf Usage }\\
  	Load the current value in the timer match1 register. Divide the value
	in half and store the new value in the timer match1 register. 
 \vskip 8pt
  \noindent

%\newpage
  \subsubsection{removeLife Flow Chart}
  \begin{center}
    \begin{tikzpicture}[node distance=2cm, scale=.50, transform shape]
      \node (0) [startstop] {start};
      \node (1) [process, below of=0] {BLINK = 10};
      \node (2) [process, below of=1] {LIVES -= 1};
      \node (3) [decision, below of=2, yshift=-1cm] {LIVES = 3?};
      \node (4) [decision, below of=3, yshift=-1cm] {LIVES = 2?};
      \node (5) [decision, below of=4, yshift=-1cm] {LIVES = 1?};
      \node (6) [process, below of=5, yshift=-1cm] {Zero LEDs on};
      \node (7) [process, below of=6] {GameOver};
      \node (31) [process, right of=3, xshift=4cm] {3 LEDs on};
      \node (32) [process, right of=4, xshift=4cm] {2 LEDs on};
      \node (33) [process, right of=5, xshift=4cm] {1 LEDs on};
      \node (8) [startstop, below of=7] {Stop};

      \draw [arrow] (0) -- (1);
      \draw [arrow] (1) -- (2);
      \draw [arrow] (2) -- (3);
      \draw [arrow] (6) -- (7);
      \draw [arrow] (7) -- (8);
     \draw [arrow] (3) -- node [anchor=west] {no} (4);
     \draw [arrow] (4) -- node [anchor=west] {no} (5);
     \draw [arrow] (5) -- node [anchor=west] {no} (6);
      \draw [arrow] (3) -- node [anchor=south] {yes} (31);
      \draw [arrow] (4) -- node [anchor=south] {yes} (32);
      \draw [arrow] (5) -- node [anchor=south] {yes} (33);
      \draw [arrow] (31) -- ([xshift=1cm]31.east) |- (8);
      \draw [arrow] (32) -- ([xshift=1cm]32.east) |- (8);
      \draw [arrow] (33) -- ([xshift=1cm]33.east) |- (8);
    \end{tikzpicture}
  \end{center}

\newpage
  \subsection{FIQ\_Handler Routine}
   \indent
	Code that handles the fast interrupts for Edge Sensitive Push Button EINT1, 
	keyboard UART0, Timer0, and Timer1.	
   \vskip 8pt
   \noindent
   {\bf Usage }\\
  	If the push button was pressed, the interrupt is cleared, and the pauseGame 
	subroutine is called. If Timer0 caused the interrupt, it is cleared and
	handleMoves subroutine is called. Timer1 interrupt is a the two minute timer
	which calls gameOver. UART interrupts are automatically cleared. Quit is called
 	if the space bar was pressed. NewGame is called if the 'g' was pressed. The 'a',
	'w', 's', and 'd' keys set Q\_DIRECTION. After all interrupts are handled both
	timer0 and time1 are re-enabled and FIQ\_Exit is called.    
 \vskip 8pt
  \noindent

%\newpage
  \subsubsection{FIQ\_Handler Flow Chart}
  \begin{center}
    \begin{tikzpicture}[node distance=2cm, scale=.50, transform shape]
      \node (0) [startstop] {start};
      \node (1) [decision, below of=0, yshift=-1cm] {EINT1 interrupt?};
      \node (2) [decision, below of=1, yshift=-2cm] {Timer0 interrupt?};
      \node (3) [decision, below of=2, yshift=-2cm] {Timer1 interrupt?};
      \node (4) [decision, below of=3, yshift=-2cm] {UART interrupt?};
      \node (5) [decision, below of=4, yshift=-2cm] {Space bar?};
      \node (6) [decision, below of=5, yshift=-1cm] {'g'?};
      \node (7) [decision, below of=6, yshift=-2cm] {IS\_GAMEOVER\_SCREEN set?};
      \node (8) [decision, below of=7, yshift=-2cm] {'w'?};
      \node (9) [decision, below of=8, yshift=-1cm] {'a'?};
      \node (10) [decision, below of=9, yshift=-1cm] {'s'?};
      \node (11) [decision, below of=10, yshift=-1cm] {'d'?};
      \node (12) [process, below of=11] {Enable Timer0, and Timer1};
      \node (13) [process, below of=12] {FIQ\_Exit};
      \node (31) [process, right of=1, xshift=3cm] {Clear interrupt};
      \node (32) [process, below of=31] {pauseGame};
      \node (33) [process, right of=2, xshift=3cm] {Clear interrupt};
      \node (34) [process, below of=33] {handleMoves};
      \node (35) [process, right of=3, xshift=3cm] {Clear interrupt};
      \node (36) [process, below of=35] {gameOver};
      \node (37) [process, right of=5, xshift=3cm] {quit};
      \node (38) [process, right of=6, xshift=3cm] {newGame};
      \node (39) [process, right of=8, xshift=3cm] {Q\_DIRECTION = 1};
      \node (40) [process, right of=9, xshift=3cm] {Q\_DIRECTION = 2};
      \node (41) [process, right of=10, xshift=3cm] {Q\_DIRECTION = 3};
      \node (42) [process, right of=11, xshift=3cm] {Q\_DIRECTION = 4};
      \node (14) [startstop, below of=13] {Stop};

      \draw [arrow] (0) -- (1);
      \draw [arrow] (31) -- (32);
      \draw [arrow] (33) -- (34);
      \draw [arrow] (35) -- (36);
      \draw [arrow] (12) -- (13);
      \draw [arrow] (13) -- (14);
     \draw [arrow] (1) -- node [anchor=west] {no} (2);
     \draw [arrow] (2) -- node [anchor=west] {no} (3);
     \draw [arrow] (3) -- node [anchor=west] {no} (4);
     \draw [arrow] (4) -- node [anchor=west] {yes} (5);
     \draw [arrow] (5) -- node [anchor=west] {no} (6);
     \draw [arrow] (6) -- node [anchor=west] {no} (7);
     \draw [arrow] (7) -- node [anchor=west] {no} (8);
     \draw [arrow] (8) -- node [anchor=west] {no} (9);
     \draw [arrow] (9) -- node [anchor=west] {no} (10);
     \draw [arrow] (10) -- node [anchor=west] {no} (11);
     \draw [arrow] (1) -- node [anchor=south] {yes} (31);
     \draw [arrow] (2) -- node [anchor=south] {yes} (33);
     \draw [arrow] (3) -- node [anchor=south] {yes} (35);
     \draw [arrow] (5) -- node [anchor=south] {yes} (37);
     \draw [arrow] (6) -- node [anchor=south] {yes} (38);
     \draw [arrow] (8) -- node [anchor=south] {yes} (39);
     \draw [arrow] (9) -- node [anchor=south] {yes} (40);
     \draw [arrow] (10) -- node [anchor=south] {yes} (41);
     \draw [arrow] (11) -- node [anchor=south] {yes} (42);
      \draw [arrow] (32) -- ([xshift=1cm]32.east) |- (12);
      \draw [arrow] (34) -- ([xshift=1cm]34.east) |- (12);
      \draw [arrow] (36) -- ([xshift=1cm]36.east) |- (12);
      \draw [arrow] (4) -- ([xshift=6cm]4.east) |- (12);
      \draw [arrow] (38) -- ([xshift=1cm]38.east) |- (12);
      \draw [arrow] (39) -- ([xshift=1cm]39.east) |- (12);
      \draw [arrow] (40) -- ([xshift=1cm]40.east) |- (12);
      \draw [arrow] (41) -- ([xshift=1cm]41.east) |- (12);
      \draw [arrow] (42) -- ([xshift=1cm]42.east) |- (12);
      \draw [arrow] (7) -- ([xshift=6cm]7.east) |- (12);
    \end{tikzpicture}
  \end{center}


\newpage
  \subsection{pauseGame Routine}
   \indent
	This routine pauses, or unpauses the game when the EINT1 push button
	is pressed. No keyboard interrupts are handled when the game is paused.
   \vskip 8pt
   \noindent
   {\bf Usage }\\
  	IS\_PAUSED is loaded and checked to see if the game needs to be
	paused, or unpaused. Whn paused, IS\_PAUSED is set, the RGB is set to blue, "PAUSE" is
	displayed on puTTY, and both timers are stopped. Unpausing the game
	sets the RGB back to green, clears IS\_PAUSED writes spaces over the 
	"PAUSED" message on puTTY, and renanbles both timers.
 \vskip 8pt
  \noindent

%\newpage
  \subsubsection{pauseGame Flow Chart}
  \begin{center}
    \begin{tikzpicture}[node distance=2cm, scale=.50, transform shape]
      \node (0) [startstop] {start};
      \node (1) [decision, below of=0, yshift=-1cm] {IS\_PAUSED set?};
      \node (2) [process, below of=1, yshift=-2cm] {Clear IS\_PAUSED};
      \node (3) [process, below of=2] {Set RGB to green};
      \node (4) [process, below of=3] {Write spaces over "PAUSED" on puTTY};
      \node (5) [process, below of=4] {Re-enable Timer0, and Timer1};
      \node (31) [process, right of=1, xshift=4cm] {Disable Timer0, and Timer1};
      \node (32) [process, below of=31] {Set IS\_PAUSED};
      \node (33) [process, below of=32] {Set RGB to blue};
      \node (34) [process, below of=33] {Write "PAUSED" to puTTY screen};
      \node (6) [startstop, below of=5] {Stop};

      \draw [arrow] (0) -- (1);
      \draw [arrow] (2) -- (3);
      \draw [arrow] (3) -- (4);
      \draw [arrow] (4) -- (5);
      \draw [arrow] (5) -- (6);
      \draw [arrow] (31) -- (32);
      \draw [arrow] (32) -- (33);
      \draw [arrow] (33) -- (34);
     \draw [arrow] (1) -- node [anchor=south] {no} (31);
     \draw [arrow] (1) -- node [anchor=west] {yes} (2);
      \draw [arrow] (34) |- (6);
    \end{tikzpicture}
  \end{center}


\newpage
  \subsection{handleMoves Routine}
   \indent
	This routine determines when enemies should be spawned and when they should be moved.
	It also controls how often the player can move Q'bert. Curses are also removed by 
	from this subroutine. blinkOnDeath is called from here and will turn RGB red or off
	if the player recently died.
   \vskip 8pt
   \noindent
   {\bf Usage }\\
	If IS\_GAMEOVER\_SCREEN is set then then no moves need to be handled, and routine 
	goes straight to FIQ\_Exit. If the game has not been running for at least two
	seconds then no enemies will be spawned. If there are aleady three enemies on the
	pyramid then no more need to be spawned. ALl currently enemies are moved. Q'bert
	is allowed another move. blinkOnDeath is called and RGB blinks red if a life was
	recently lost.
 \vskip 8pt
  \noindent

%\newpage
  \subsubsection{handleMoves Flow Chart}
  \begin{center}
    \begin{tikzpicture}[node distance=2cm, scale=.50, transform shape]
      \node (0) [startstop] {start};
      \node (1) [decision, below of=0, yshift=-4cm] {IS\_GAMEOVER\_SCREEN set?};
      \node (2) [decision, below of=1, yshift=-4cm] {Is NUM\_HALF\_SECS > 4?};
      \node (3) [decision, below of=2, yshift=-4cm] {Is NUM\_BALLS < 3?};
      \node (4) [process, below of=3, yshift=-2cm] {moveBall1};
      \node (5) [process, below of=4] {moveBall2};
      \node (6) [process, below of=5] {moveSnakeBall};
      \node (7) [process, below of=6] {moveSnake};
      \node (8) [process, below of=7] {Q\_MOVES = 1};
      \node (9) [process, below of=8] {blinkOnDeath};
      \node (10) [decision, below of=9, yshift=-2cm] {DISPLAY\_CURSE set?};
      \node (11) [process, below of=10, yshift=-4cm] {FIQ\_Exit};
      \node (31) [process, right of=3, xshift=4cm] {spawnEnemy};
      \node (32) [process, right of=10, xshift=4cm] {clearCurse};
      \node (12) [startstop, below of=11] {Stop};

      \draw [arrow] (0) -- (1);
     \draw [arrow] (1) -- node [anchor=west] {no} (2);
     \draw [arrow] (2) -- node [anchor=west] {yes} (3);
     \draw [arrow] (3) -- node [anchor=west] {no} (4);
     \draw [arrow] (10) -- node [anchor=west] {no} (11);
      \draw [arrow] (4) -- (5);
      \draw [arrow] (5) -- (6);
      \draw [arrow] (6) -- (7);
      \draw [arrow] (7) -- (8);
      \draw [arrow] (8) -- (9);
      \draw [arrow] (9) -- (10);
      \draw [arrow] (3) -- (31);
     \draw [arrow] (3) -- node [anchor=south] {yes} (31);
      \draw [arrow] (11) -- (12);
     \draw [arrow] (31) |- (4);
     \draw [arrow] (32) |- (12);
     \draw [arrow] (10) -- node [anchor=south] {yes} (32);
      \draw [arrow] (1) -- node [anchor=south] {yes}([xshift=6cm]1.east) |- (12);
      \draw [arrow] (2) -- node [anchor=south] {no}([xshift=6cm]2.east) |- (12);
    \end{tikzpicture}
  \end{center}


\newpage
  \subsection{spawnEnemy Routine}
   \indent
	Spawns a new enemy depending on which enemies are already spawned as well
	as random numbers.
   \vskip 8pt
   \noindent
   {\bf Usage }\\
	If a snake or snakeBall already occupy a square on the pyramid, a snakeball
	will not spawn. If they are not already spawned there is a one in four
	chance to spawn one. Otherwise if there is no ball1 then ball1 will spawn.
	if there is already a ball1, then a ball2 will spawn.
 \vskip 8pt
  \noindent

%\newpage
  \subsubsection{spawnEnemy Flow Chart}
  \begin{center}
    \begin{tikzpicture}[node distance=2cm, scale=.50, transform shape]
      \node (0) [startstop] {start};
      \node (1) [decision, below of=0, yshift=-4cm] {SNAKE\_SQUARE OR SNAKEBALL\_SQUARE !=0?};
      \node (3) [decision, below of=2, yshift=-1cm] {Is number = 4?};
      \node (4) [decision, below of=3, yshift=-4cm] {BALL1\_SQUARE = 0?};
      \node (2) [process, below of=1, yshift=-4cm] {random number 0-3};
      \node (5) [process, below of=4, yshift=-2cm] {spawn ball1};
      \node (31) [process, right of=3, xshift=4cm] {spawn snakeBall};
      \node (32) [process, right of=4, xshift=4cm] {spawn ball2};
      \node (6) [startstop, below of=5] {Stop};

      \draw [arrow] (0) -- (1);
     \draw [arrow] (1) -- node [anchor=west] {yes} (2);
      \draw [arrow] (1) -- node [anchor=south] {no}([xshift=-2cm]1.west) |- (4);
     \draw [arrow] (3) -- node [anchor=south] {yes} (31);
     \draw [arrow] (3) -- node [anchor=west] {no} (4);
     \draw [arrow] (4) -- node [anchor=south] {no} (32);
      \draw [arrow] (2) -- (3);
      \draw [arrow] (5) -- (6);
     \draw [arrow] (4) -- node [anchor=west] {yes} (5);
     \draw [arrow] (32) |- (6);
    \end{tikzpicture}
  \end{center}



\newpage
  \subsection{moveBall1 Routine}
   \indent
	Moves ball1 randomly down,or right.
   \vskip 8pt
   \noindent
   {\bf Usage }\\
	Get a random number between 0-1. If it is a 0, move down, otherwise
	move right. Remove the 'o'. Check if ball1 has moved off the pyramid. 
	If not, draw 'o' in new square. 
 \vskip 8pt
  \noindent

%\newpage
  \subsubsection{moveBall1 Flow Chart}
  \begin{center}
    \begin{tikzpicture}[node distance=2cm, scale=.50, transform shape]
      \node (0) [startstop] {start};
      \node (1) [process, below of=0] {random number 0-1};
      \node (2) [decision, below of=1, yshift=-1cm] {Number = 1?};
      \node (3) [process, below of=2, yshift=-1cm] {Move down};
      \node (4) [process, below of=3] {remove 'o'};
      \node (5) [decision, below of=4, yshift=-1cm] {Ball1 off pyramid?};
      \node (6) [process, below of=5, yshift=-1cm] {draw 'o'};
      \node (31) [process, right of=2, xshift=4cm] {Move right};
      \node (7) [startstop, below of=6] {Stop};

      \draw [arrow] (0) -- (1);
      \draw [arrow] (1) -- (2);
     \draw [arrow] (2) -- node [anchor=south] {yes} (31);
      \draw [arrow] (31) -- ([yshift=-2cm]31.south) |- (5.north);
     \draw [arrow] (2) -- node [anchor=west] {no} (3);
      \draw [arrow] (3) -- (4);
      \draw [arrow] (4) -- (5);
     \draw [arrow] (5) -- node [anchor=south] {yes}([xshift=2cm]5.east) |- (7);
     \draw [arrow] (5) -- node [anchor=west] {no} (6);
     \draw [arrow] (6) -- (7);
    \end{tikzpicture}
  \end{center}


\newpage
  \subsection{moveBall2 Routine}
   \indent
	Moves ball2 randomly down,or right.
   \vskip 8pt
   \noindent
   {\bf Usage }\\
	Get a random number between 0-1. If it is a 0, move down, otherwise
	move right. Remove the 'o'. Check if ball2 has moved off the pyramid. 
	If not, draw 'o' in new square. 
 \vskip 8pt
  \noindent

%\newpage
  \subsubsection{moveBall2 Flow Chart}
  \begin{center}
    \begin{tikzpicture}[node distance=2cm, scale=.50, transform shape]
      \node (0) [startstop] {start};
      \node (1) [process, below of=0] {random number 0-1};
      \node (2) [decision, below of=1, yshift=-1cm] {Number = 1?};
      \node (3) [process, below of=2, yshift=-1cm] {Move down};
      \node (4) [process, below of=3] {remove 'o'};
      \node (5) [decision, below of=4, yshift=-1cm] {Ball2 off pyramid?};
      \node (6) [process, below of=5, yshift=-1cm] {draw 'o'};
      \node (31) [process, right of=2, xshift=4cm] {Move right};
      \node (7) [startstop, below of=6] {Stop};

      \draw [arrow] (0) -- (1);
      \draw [arrow] (1) -- (2);
     \draw [arrow] (2) -- node [anchor=south] {yes} (31);
     %\draw [arrow] (31) -| (5.north);
      \draw [arrow] (31) -- ([yshift=-2cm]31.south) |- (5.north);
     \draw [arrow] (2) -- node [anchor=west] {no} (3);
      \draw [arrow] (3) -- (4);
      \draw [arrow] (4) -- (5);
     \draw [arrow] (5) -- node [anchor=south] {yes}([xshift=2cm]5.east) |- (7);
     \draw [arrow] (5) -- node [anchor=west] {no} (6);
     \draw [arrow] (6) -- (7);
    \end{tikzpicture}
  \end{center}


\newpage
  \subsection{moveSnakeBall Routine}
   \indent
	Moves snakeBall randomly down,or right. Spawns a snake when it reaches
	bottom of pyramid
   \vskip 8pt
   \noindent
   {\bf Usage }\\
	Get a random number between 0-1. If it is a 0, move down, otherwise
	move right. Remove the 'C'. Check if the snakeBall has moved off the pyramid. 
	If not, draw 'C' in new square. If it has moved off the pyramid replace the
	'C' in the bottom square with a 'S' for snake enemy.
	 
 \vskip 8pt
  \noindent

%\newpage
  \subsubsection{moveSnakeBall Flow Chart}
  \begin{center}
    \begin{tikzpicture}[node distance=2cm, scale=.50, transform shape]
      \node (0) [startstop] {start};
      \node (1) [process, below of=0] {random number 0-1};
      \node (2) [decision, below of=1, yshift=-1cm] {Number = 1?};
      \node (3) [process, below of=2, yshift=-1cm] {Move down};
      \node (4) [process, below of=3] {remove 'C'};
      \node (5) [decision, below of=4, yshift=-3cm] {snakeBall off pyramid?};
      \node (6) [process, below of=5, yshift=-3cm] {draw 'C'};
      \node (31) [process, right of=2, xshift=4cm] {Move right};
      \node (32) [process, right of=5, xshift=6cm] {SNAKE\_SQUARE = SNAKEBALL\_SQUARE};
      \node (33) [process, below of=32] {Draw 'S'};
      \node (7) [startstop, below of=6] {Stop};

      \draw [arrow] (0) -- (1);
      \draw [arrow] (1) -- (2);
     \draw [arrow] (2) -- node [anchor=south] {yes} (31);
      \draw [arrow] (31) -- ([yshift=-2cm]31.south) |- (5.north);
     \draw [arrow] (2) -- node [anchor=west] {no} (3);
      \draw [arrow] (3) -- (4);
      \draw [arrow] (4) -- (5);
     \draw [arrow] (5) -- node [anchor=south] {yes} (32);
      \draw [arrow] (32) -- (33);
      \draw [arrow] (33) |- (7);
     \draw [arrow] (5) -- node [anchor=west] {no} (6);
     \draw [arrow] (6) -- (7);
    \end{tikzpicture}
  \end{center}


\newpage
  \subsection{moveSnake Routine}
   \indent
	Moves the snake in the direction of Q'bert on the pyramid. Only one snake 
	will be spawned at a time. 
   \vskip 8pt
   \noindent
   {\bf Usage }\\
	If the current Q\_SQUARE is above the SNAKE\_SQUARE move the snake up.
	If the current Q\_SQUARE is below the SNAKE\_SQUARE move the snake down.
	If the current Q\_SQUARE is left of the SNAKE\_SQUARE move the snake left.
	If the current Q\_SQUARE is right of the SNAKE\_SQUARE move the snake right.
 \vskip 8pt
  \noindent

%\newpage
  \subsubsection{moveSnake Flow Chart}
  \begin{center}
    \begin{tikzpicture}[node distance=2cm, scale=.50, transform shape]
      \node (0) [startstop] {start};
      \node (1) [decision, below of=0, yshift=-3cm] {Q\_SQUARE above SNAKE\_SQUARE?};
      \node (2) [decision, below of=1, yshift=-6cm] {Q\_SQUARE below SNAKE\_SQUARE??};
      \node (3) [decision, below of=2, yshift=-6cm] {Q\_SQUARE left of SNAKE\_SQUARE?};
      \node (4) [process, below of=3, yshift=-3cm] {move right};
      \node (31) [process, right of=1, xshift=5cm] {move up};
      \node (32) [process, right of=2, xshift=5cm] {move down};
      \node (33) [process, right of=3, xshift=5cm] {move left};
      \node (5) [startstop, below of=4] {Stop};

      \draw [arrow] (0) -- (1);
     \draw [arrow] (1) -- node [anchor=south] {yes} (31);
     \draw [arrow] (2) -- node [anchor=south] {yes} (32);
     \draw [arrow] (3) -- node [anchor=south] {yes} (33);
      \draw [arrow] (31) -- ([xshift=1cm]31.east) |- (5);
      \draw [arrow] (32) -- ([xshift=1cm]32.east) |- (5);
      \draw [arrow] (33) -- ([xshift=1cm]33.east) |- (5);
     \draw [arrow] (1) -- node [anchor=west] {no} (2);
     \draw [arrow] (2) -- node [anchor=west] {no} (3);
     \draw [arrow] (3) -- node [anchor=west] {no} (4);
      \draw [arrow] (4) -- (5);
    \end{tikzpicture}
  \end{center}


\newpage
  \subsection{randomNum Routine}
   \indent
	This routine take a number and generates a random number from zero through
	the given number minus one.
   \vskip 8pt
   \noindent
   {\bf Usage }\\
	This routine gets the time from timer1 clocc register. This number is added to 
	the number of lives the player has left, the squares that ball1 and ball2
	are currently on. This total is then divided by the given number to generate
	a remaineder of zero up to the number given minus one(i.e. if given the number
	4 a number between 0-3 will be returned).
 \vskip 8pt
  \noindent

%\newpage
  \subsubsection{randomNum Flow Chart}
  \begin{center}
    \begin{tikzpicture}[node distance=2cm, scale=.50, transform shape]
        \node (0) [startstop] {start};
	\node (1) [process, below of=0] {y = number of randoms required};
	\node (2) [process, below of=1] {x = get timer1 clock time};
	\node (3) [process, below of=2] {x += Q\_SQUARE};
	\node (4) [process, below of=3] {x += LIVES};
	\node (5) [process, below of=4] {x += BALL1\_SQUARE};
	\node (6) [process, below of=5] {x += BALL2\_SQUARE};
	\node (7) [process, below of=6] {x div\_and\_mod y};
	\node (8) [process, below of=7] {random number = remainder};
        \node (9) [startstop, below of=8, yshift=-1cm] {Stop};
    
        \draw[arrow] (0) -- (1);
        \draw[arrow] (1) -- (2);
        \draw[arrow] (2) -- (3);
        \draw[arrow] (3) -- (4);
        \draw[arrow] (4) -- (5);
        \draw[arrow] (5) -- (6);
        \draw[arrow] (6) -- (7);
        \draw[arrow] (7) -- (8);
        \draw[arrow] (8) -- (9);
    \end{tikzpicture}
  \end{center}


%end the document
\end{document}
